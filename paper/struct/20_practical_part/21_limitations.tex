\subsubsection{De-duplication}
<<...>>

\subsubsection{Computing Hardware}
<<...>>

\subsubsection{Data}
\textbf{Disparity in Access to Financial Resources.} During the study, it was found that there exists a significant barrier for individual
researchers who lack the financial resources required for expensive data collection, infrastructure rental, and the time needed to develop
a system entirely from scratch.

The financial community — which includes news outlets, data aggregators, professional traders, and investment funds — often does not facilitate
the development of publicly available tools for extracting value from financial instruments. On the contrary, several market participants deliberately
create additional obstacles to free data access, while failing to utilize existing resources efficiently. Examples include:

\begin{itemize}
    \item \textbf{Infrastructure limitations.} Restrictions imposed by aggregators and news services (e.g., Yahoo! Finance) impede large-scale data collection.
    \item \textbf{Closed APIs and high tariffs.} Services such as Google Finance and Yahoo! Finance, along with platforms like Twitter and Seeking Alpha,
    offer limited functionality or charge high fees for access.
    \item \textbf{Restrictions on access to analytical tools.} Cases such as BloombergGPT illustrate the deliberate concealment of general-purpose tools.
    \item \textbf{Strict copyright policies.} Tighter copyright conditions result in restricted access to various datasets.
\end{itemize}

Thus, it can be concluded that the financial community contributes to a scarcity of open informational resources by artificially raising
the barriers to access with the aim of reducing competition and limiting the number of independent market players.

This issue is not new --- it has been repeatedly highlighted in several studies (including by the creators of FinBERT \parencite{Yang2020FinBERT});
however, over the past five years the situation has remained virtually unchanged. A crisis also persists in the open-source segment of financial tools.

Despite the widespread restrictive practices, there are proactive participants in the financial sector who strive to distribute information
more equitably. For instance, the financial data provider Alpha Vantage offers a free and open API that grants access to a vast array of valuable
data, including intraday OHLCV. Although Reddit is less popular than platform X (ex-Twitter) in the financial community, it also provides an open API
and can serve as an alternative channel for publishing announcements, opinions, and insider information.

In addition, aggregators such as FinURLs and MarketWatch represent important information sources. FinURLs compiles links to historical news
from 24 sources over several years. Despite the lack of a dedicated API and certain interface inconveniences for data extraction, this resource
remains valuable. At the same time, MarketWatch boasts a more advanced infrastructure by offering not only links to news articles but also quantitative
data, as well as the ability to obtain information on specific markets, assets, or indices.

Individual yet significant sources, such as the websites of certain companies and government agencies, also deserve attention. For example,
the SEC provides free access to historical financial reports (e.g., 10-K and 10-Q) via an RSS feed, thereby promoting more equitable access
to information. However, even these open datasets are frequently accompanied by technical challenges: precise timestamps are often missing
or the website structure is disrupted, which complicates automated data extraction.

It should be noted that nearly all real-time data are available without significant restrictions, as most services promptly provide such information.
Nevertheless, the collection of both historical and real-time data regularly encounters ethical and copyright issues, which remain an important aspect
in the practical use of these resources.

In summary, despite various initiatives aimed at expanding access to financial data, the overall landscape is still characterized by artificially
high barriers. These restrictions contribute to a shortage of open tools, which in turn reduces market competition and limits opportunities
for independent researchers. Therefore, the development of methodologies aimed at the free and equitable dissemination of information remains
an urgent task, requiring a comprehensive approach that takes technical, ethical, and legal aspects into account.

\subsubsection{Context (Attention Mechanism)}
<<...>>