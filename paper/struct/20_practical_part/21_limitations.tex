Prior to practical implementation, a comprehensive analytical evaluation was conducted
based on preliminary semantic analysis and structural design to identify and formalize
the technical constraints affecting the FinABYSS architecture. As a result, five key
problem areas were identified, of which three were able to offer sustainable solutions,
while two less critical ones were left for further research.

\textbf{Deduplication of texts.} De-duplication is a necessary element of the text sentiment
analysis pipeline because financial signals propagating with delay may be repeatedly fed
into the corpus, and their repeated consideration distorts prediction results. For example,
the initial publication of mortgage origination irregularities in September 2008 generated
strong negative sentiment at the time of the event's occurrence, while its republications
years later accompanying retrospective reviews do not have a similar effect on asset prices.
Such discreteness of temporal context is not accounted for in classical textual deduplication
based on linguistic or syntactic similarity.

Two news items can have almost complete textual similarity and yet carry dramatically different
semantic connotations. As a theoretical example, if in the original article about the Citi Group
fine (Section \ref{sec:practical_importance}), “79 million” had been mistakenly replaced with
“79 thousands”, it would have had a fundamentally different sentiment and a different meaning.

A special category of duplicate materials is represented by corrections on the Yahoo! Finance
platform. Such publications begin with the marker “/CORRECTION/”, then the article points
to the corrected fragments and repeats the main text almost verbatim. Linear de-duplication
at the line or $n$-gram level in such cases removes the latest version, which violates
the logic of stream analysis.

Thus, de-duplication in a financial media stream should fulfill two requirements: first,
to distinguish semantically equivalent but contextually different publications in time;
second, to correctly handle “correction”-versions, extracting non-overlapping semantics
if it differs meaningfully from the previous version. Each document should be considered
unique if it contains new information, even if the textual overlap rate is high.

Since the problem lacks source markup, it requires a formal formulation of the semantic
deduplication problem that goes beyond simple textual comparison. The semantic space
of texts, unlike the lexical space, is continuous and unbounded, and therefore pairwise
comparison of strings or texts by cosine distance is not sufficient to check the uniqueness
of a single document: it is necessary to operate with volumes of vector representations and
to take into account the distribution of ideas in a wider context.

A full description of the mathematical formulation of the problem and the proposed analytical
solution is presented in Section \ref{sec:semantic_deduplication}. It is there that a strict
definition of semantic uniqueness is given, and an algorithm for detecting documents that
are close in meaning but differ in information value is presented.

\textbf{Computing Resources.} Training models on big data requires an extremely powerful computing
infrastructure. For example, one FinBERT model was trained on four NVIDIA Tesla P100s for two
days on 5 billion tokens \parencite{Yang2020FinBERT}. It should be clarified that the training time of only one
model is given here, and several more models are usually trained during the experimentation
process. Speaking of ModernBERT --- it was trained on eight NVIDIA H100s in 10 days, and the total
corpus was about 3 trillion tokens \parencite{Warner2024ModernBERT}.

In the context of our study, ModernBERT is used as the baseline model, so we should focus on it.
In the course of the study, a corpus of about 1 billion tokens was collected, which, although smaller
than the ModernBERT corpus, would be sufficient for domain-adapted pre-training, i.e., an additional
fourth stage of pre-training \parencite{Warner2024ModernBERT,gururangan2020DAPT}.

Training even the basic ModernBERT model requires extremely powerful computing hardware. The minimum
requirements are NVIDIA Tesla T4 from the server segment or NVIDIA RTX 3090 from the customer segment
\parencite{Warner2024ModernBERT}. However, in the context of the study, only NVIDIA RTX 3060 was available,
on which it would have been impossible to adapt ModernBERT. Therefore, due to the difficult availability
of computational resources, the context of the study uses a baseline version of ModernBERT fine-tuned
for the task of clustering vector representations.

\textbf{Data.} The lack of a representative text corpus in the financial domain was one of the key challenges
of this study. Attempts to find open-source datasets on HuggingFace, Kaggle, GitHub and similar platforms
revealed that the available financial corpora have either been removed for copyright infringement reasons,
are closed for public use, or are too short fragments of text unsuitable for topic modeling and sentiment
analysis in the current long-context research field
\parencite{FiQA2018SA, Malo2014FPB, daudert2022multi, xing2020financial, wiebe2005annotating}.
In addition, many of the remaining sets are solely for fine-tuning neural network models, but do not
contain the necessary amount or variety of data.

Self-collection of financial news articles is complicated by the fact that key sources are disparate and often
resist mass scraping. Few providers provide access to paid and expensive APIs (e.g., X, formerly Twitter),
where prices for historical data can run into the thousands of dollars. Meanwhile, news sites tend to focus
on narrow segments of industry content, making single-source corpus biased and fragmented.

From a financial point of view, not only the breadth of coverage is important, but also the availability
of reliable metadata: timestamps, information about the author and others. Many sites lack open archives,
provide only RSS feeds, and some resources simply do not place publication time in HTML, which makes automated
parsing impossible without in-depth analysis and regular script updates. Different HTML template structures
and dynamic content generators (JavaScript rendering) increase the complexity of developing data extraction
pipelines.

Thus, trying to assemble a truly representative corpus would require integrating multiple sources
with heterogeneous site layouts, resulting in a high technical burden. Focusing on a single data provider
risks introducing an imbalance in thematic and regional representations of financial events. At the same
time, existing paid alternatives (e.g., commercial datasets of News from Bright
Data\footnote{URL: \url{https://brightdata.com/products/datasets/news}}) are priced at several thousand
dollars for a volume comparable to the corpus collected during the study, which is beyond the budget
and research constraints of the project.

Consequently, the lack of open, large and homogeneous financial corpora remains a serious obstacle for building
scalable and robust models for topic modeling and tone analysis in finance. The following Section \ref{sec:data_collecting}
describes the chosen compromise approach to data collection and aggregation, taking into account the identified
limitations and quality requirements of the corpus.

\textbf{Context window limitation.} The base version of the ModernBERT model provides a context window of 8,192 tokens,
which significantly outperforms traditional BERT-like models with a 512 token limit \parencite{devlin2019BERT,Warner2024ModernBERT}. However,
even with this extended resource, 10-K and 10-Q financial documents, often exceeding tens of pages, do not fit
completely. While there are techniques for sliding windowing or fragmenting text into overlapping chunks, they
are beyond the scope of the original task of evaluating the out-of-the-box capabilities of modern LLMs.
In the context of this study, it was decided to limit the analysis to news articles whose sizes fit within 8,192
tokens, and not to consider multi-page analytical reports. This allowed us to maintain the focus on comparing
embodiment and topic models without complicating the preprocessing by aggregating excerpts of long documents.

\textbf{The irrelevance of existing benchmarks.} Comparing the performance of the ModernBERT model and its
domain-adapted version on the widely accepted FLUE benchmark seems like a natural step for evaluating progress.
However, FLUE datasets consist mostly of short chunks (up to 512 tokens) designed for typical text comprehension
tasks \parencite{FLANG2022FLUE}. Because these datasets are sharpened for the 512-token window, they do not
reflect the benefits of ModernBERT's extended context and, conversely, will underestimate its totals. Thus,
the use of FLUE in the current study will lead to a distorted perception of the model's qualities: short-text
tasks do not demonstrate its ability to capture long-term dependencies and synthesize information from large
amounts of data.

Both problems --- the limited context window when analyzing long documents and the unrepresentativeness of standard
512-token benchmarks --- dictate the need for specialized preprocessing and validation techniques tailored
to the financial domain. With the exception of these two problems, all other problems have been solved and their
solutions are proposed in further sections of the paper.