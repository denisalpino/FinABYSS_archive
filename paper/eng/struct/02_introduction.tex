In recent years, the use of data analysis and artificial intelligence, in particular machine Learning (ML)
and deep Learning (DL), to make investment decisions has become an integral part of the strategies of many
traders, companies and funds. However, modern financial markets are characterized by high volatility
and high speed of information dissemination, which creates significant difficulties for analyzing its impact
on stock prices.

Events such as breaking news releases, legislative changes, and analyst reviews can have an impact.
both direct and cumulative effects on market performance. However, traditional approaches to
data analysis often ignore the dynamics of these impacts, which leads to low forecast accuracy and, as a result,
ineffective investment strategies.

To compete in a rapidly changing financial environment, companies need to continually optimize their
approaches to data analysis. This requires the development of tools that can not only take into account the dynamic
the nature of the information, but also to provide forecasts based on an in-depth analysis of events and their cumulative effect.
This article responds to these challenges by offering a methodology and technological solution for deep, interoperable
and simple financial analytics based on deep neural networks, namely large language models (LLM).

The effectiveness of classical ML algorithms in financial forecasting has been proven in many studies.
However, DL and natural language processing (NLP) architectures literally changed the rules of the game after the advent of
transformer architecture in 2017 \parencite{vaswani2017attention}. Since then, LLMs have become widespread
and have proven their applicability in various applications, including asset value forecasting.
\parencite{Jiang2023, Halder2022, Kim2023}.

Modern research demonstrates the high efficiency of LLM for solving a number of tasks related to asset valuation and
forecasting. Nevertheless, the issues of integrating LLM with classical quantitative methods remain unresolved.
models, the lack of a sufficient number of open source solutions for the financial field, and the limitations
of existing models for processing long text sequences (see \hyperref[sec:models]{Section 1.3.1}).
In December 2024, a new state-of-the-art ModernBERT model was introduced, capable of processing texts
16 times longer than the previous ones architectures \parencite{Warner2024ModernBERT, devlin2019BERT}. This model
expands the analysis capabilities from a single headline or post to entire news articles, press releases, and transcripts.
interviews and analytical reviews. Despite this, processing full-fledged financial statements (for example, 10-Q, 10-K)
remains a difficult task (see \hyperref[sec:limits]{Section 2.1}). At the same time, the emphasis on analyzing full articles
avoids the problems associated with the clickbait effect and the lack of context for decision-making.

Language models, pre-trained on common text corpora, do not always effectively solve the problems of financial forecasting.
\parencite{Jiang2023}. This is due to the uniqueness of financial information and the richness of special terminology.
and jargon, which makes it difficult to apply general-purpose models. In this regard, there is a need to adapt
the basic LLM for the financial sector.

Thus, if we consider the problem from the management side, when making investment decisions in volatile markets,
it is important to timely analyze the cumulative impact of various events (news, legislative changes, analytics, etc.)
on asset dynamics. Specialists are not able to cope with such a volume of information in the shortest possible time. With another
On the other hand, the lack of a comprehensive analysis tool leads to delayed or inaccurate decisions, which reduces
the effectiveness of investment strategies and increases the risk of missing profitable opportunities.

Developing a tool that could resolve this situation and simplify the decision-making process is a time-consuming
and complex task that can be roughly divided into the following stages:

\begin{enumerate}
    \item Development of an effective architecture for LLM.
    \item Adaptation of the model to the specifics of the financial domain.
    \item Fine-tuning the model to solve specific tasks.
    \item Integration of the model into a system that works with both quantitative and qualitative data,
    including the stages of training, testing, and implementation.
\end{enumerate}

It is important to note that the ultimate goal of such a tool is autonomous effective financial forecasting, which
It could easily be interpreted by a financial analyst who oversees the system. It is worth clarifying that the speech
We are talking about effective financial forecasting precisely from the point of view of the theory of market efficiency (EMT) \parencite{emt1970fama}.

So, since the basic architecture of ModernBERT has already been developed, and methods for predicting financial assets using neural
networks have been sufficiently studied, this study focuses on developing tools to improve both the efficiency and
interpretability of financial forecasting. Moreover, the study formulates the goal to achieve increased efficiency forecasting,
not by quantifying or iteratively improving existing methods, but by exploring fundamentally new areas and making a significant
technological leap in the new paradigm of financial forecasting. So, the present study suggests
achieving a qualitative leap by improving the data used for forecasting.

This task, as noted earlier, is complex and extremely difficult. Therefore, the current work does not focus on financial forecasting itself.,
It focuses on the development of tools, technology and a set of methods that will make possible a qualitative breakthrough
in the target task in subsequent research.

So, at the moment, LLMs are used in an extremely narrow range of financial forecasting, since the technology itself is relatively new.
Nevertheless, there are already the first attempts to use language models to analyze the tonality of news and integrate tonalities into
the forecasting process, and despite limited resources, methods and technologies, this approach has proven to be quite effective.
effective. In particular, models that use tonalities in financial forecasting show results higher than those that make
forecasts based solely on quantitative data \parencite{Kim2023, Jiang2023, Halder2022}. However, the applicability of the textual
modality for financial forecasting is currently extremely limited, and a new direction that has not yet found application
in finance is aspect-oriented sentimental analysis \parencite{SA2020taxonomy,FSA2020problems} and thematic modeling
\parencite{angelov2020top2vec,BERTopic2022}.

Thus, the object of the research is investment strategies based on artificial intelligence methods, and the subject is a
toolkit for integrating language models, as well as methods of thematic modeling and tonality analysis into the process
of asset value forecasting. The purpose of the work is to create a practice-oriented interpretable toolkit, on the basis of which it will
be possible to develop a system for dynamic multimodal forecasting using a hierarchical aspect-oriented approach.
sentimental analysis. It is worth emphasizing that the key condition of the proposed solution is its interpretability,
unlike other solutions that use deep neural networks as a black box.

The following works were performed during the study:

\begin{itemize}
    \item Analysis and selection of modern and most efficient architectures and models (see Sections \hyperref[sec:ml_algos]{1.2} and
    \hyperref[sec:models]{1.3}) for the target task.
    \item Designing a concept for a fundamentally new system architecture for effective financial forecasting
    (see Section \hyperref[sec:architecture]{3.1}).
\item Development of a financial Aspect-Based hybrid semantic system (FinABYSS)
aimed at implementing the necessary system components for interpreted financial analytics on multimodal data
    (see Section \hyperref[sec:components]{3.2}).
\item Collection of a large and comprehensive corpus of financial publications larger than 15 GB, suitable for use in related financial research
(see Section \hyperref[sec:data_governance]{2.2}).
\end{itemize}

All the results of this study, including the code for data collection, the training code, the results of data analysis and models, as well as the models themselves are available
in the official GitHub repository of the project.\footnote{FinABYSS (Financial Aspect-Based hybrid Semantic System) [Electronic resource] //
Tomin D.V. --- 2025 --- URL: \url{https://github.com/denisalpino/FinABYSS } (Date of access: 05/26/2025). --- Access mode: on request.}.
The assembled case is available in the HuggingFace repository\footnote{YahooFinanceNewsRaw [Electronic resource] //
Tomin D.V. --- 2025 --- URL: \url{https://huggingface.co/datasets/denisalpino/YahooFinanceNewsRaw } (Date of access: 04/20/2025). --- Access mode: on request.}.