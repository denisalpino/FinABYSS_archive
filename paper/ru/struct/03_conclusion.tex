Настоящая работа затрагивает одну из наиболее перспективных и в то же время сложных задач современной
финансовой аналитики --- интерпретируемое  прогнозирование стоимости активов с точки зрения теории
эффективного рынка на основе аспектно-ориентированного анализа новостного потока с применением
современных архитектур глубокого обучения. Стремительное развитие LLM, а также их применение в задачах
финансовой аналитики продемонстрировали существенный прогресс по сравнению с классическими подходами,
однако одновременно обозначили ряд нерешённых проблем, касающихся как архитектурных ограничений
моделей, так и дефицита специализированных данных и средств интерпретации результатов. В рамках
настоящего исследования была предпринята попытка интеграции нескольких современных парадигм (DAPT,
ABSA, DTM, гибридная кластеризация и др.) в единое аналитическое решение с учётом специфики финансового
домена и требований к прозрачности модели.

Ключевым результатом работы стало проектирование и реализация прикладной архитектуры, позволяющей
обрабатывать длинные финансовые тексты (до 8192 токенов) с учётом их семантической и аспектной структуры.
Для этого была задействована новейшая архитектура ModernBERT, обладающая рядом существенных преимуществ
по сравнению с предыдущими версиями FinBERT и BERT. Среди них — повышенное контекстное окно, ротационные
позиционные кодировки, чередующийся механизм внимания и оптимизации, ориентированные на работу с большими
документами. Это позволило обрабатывать не только заголовки или краткие выдержки, как в большинстве
предыдущих исследований, но и полные статьи, пресс-релизы, аналитические обзоры, а также трансформировать
семантику длинных текстов в векторные представления, пригодные для кластеризации и последующего анализа.

Одним из наиболее значимых вкладов данной работы является построение пайплайна тематического моделирования,
в котором аспекты рассматриваются как тематические кластеры, выделяемые в эмбеддинговом пространстве текстов
с применением методов понижения размерности (UMAP) и иерархической кластеризации на основе плотности
(HDBSCAN). Такой подход позволяет обойти необходимость ручного определения аспектов и заранее фиксированных
словарей, что особенно важно в условиях постоянно изменяющейся терминологии, лексики и повестки дня финансового
домена. В отличие от классического ABSA, в предложенном решении извлечение аспектов осуществляется на уровне
нескольколатентных тем, что делает систему адаптивной и способной масштабироваться на новые информационные потоки
без ручной разметки.

Разработанная система сочетает  взаимодополняющих компонентов: глубокую эмбеддинговую модель,
компонент тематической агрегации, кластеризацию текстов, метрики плотностной валидации и визуализатор
интерактивной семантической карты. Совместное использование этих компонентов обеспечивает не только высокую
эффективность, но и интерпретируемость, которая является важнейшим требованием в управленческой и финансовой
аналитике. Например, предложенный механизм генерации названий тем с использованием GPT-4o и ранжирования
терминов на основе c-TF-IDF и MMR, позволяет наглядно представить, за счёт каких ключевых слов формируется
та или иная тема, а также какие документы вносят в неё основной вклад.

В ходе исследования было выявлено и проанализировано несколько существенных ограничений, возникающих
в процессе построения подобной системы. В частности, отсутствие открытых специализированных корпусов
финансовых текстов привело к необходимости создания собственного корпуса новостей с Yahoo! Finance, объём
которого превысил 15 ГБ. Также был зафиксирован ряд технических и теоретических проблем: отсутствие
консенсуса по методологии семантической дедубликации, трудности с обработкой длинных документов (например,
отчётов 10-K), а также невозможность справедливого сравнения моделей с помощью существующих бенчмарков,
основанных на коротких текстах (например, FLUE). Для преодоления этих проблем были предложены авторские
решения, включая математическую формализацию дедубликации на основе семантического объёма.

Метрики оценки качества кластеризации и тематического моделирования в данной работе также рассматривались
скрупулёзно. Помимо стандартного коэффициента силуэта, был задействован индекс DBCV --- специализированная
метрика для оценки кластеров произвольной формы с учётом плотности. Результаты показали, что применение
DBCV позволяет корректно отразить качество кластеризации в задачах, где классические метрики (например,
индекс Дэвиса–Болдена, Калински–Харабаша или VIASCKDE ) дают искажённую картину.

Особое внимание в работе было уделено не только архитектурной, но и методологической строгости. Этапы
построения пайплайна были продуманы с учётом требований воспроизводимости и масштабируемости, а также
потенциальной интеграции с количественными моделями. Поскольку задача аспектного анализа --- это не только
инструмент понимания текстов, но и средство улучшения прогноза динамики стоимости активов, предлагаемое
решение нацелено на последующую интеграцию с количественными предикторами. Тем самым работа закладывает
основу для гибридных подходов к прогнозированию, где сочетаются качественные и количественные источники
информации. Настоящее исследование является своего рода мостом между новейшими открытиями в области глубокого
обучения и областью финансов.

Важным теоретическим вкладом является также обсуждение связи между тематическим моделированием и аспектным
анализом. В данной работе показано, что аспекты в финансовых текстах могут быть интерпретированы как латентные
темы, выделяемые с помощью кластеризации в эмбеддинговом пространстве. Это открывает путь к унификации двух
ранее разрозненных направлений анализа текста — тематического моделирования и аспектного анализа сентимента.
Такой подход особенно эффективен в условиях отсутствия размеченных данных и высокой изменчивости тематики
финансовых публикаций. С другой стороны, исследование предлагает новый подход, который позволяет посмотреть
на определение и саму идею тональности текстов под другим, непредвзятым углом.

Подводя итоги, можно сформулировать основные достижения работы:

\begin{enumerate}
    \item Реализована архитектура интерпретируемой системы аспектно-ориентированного анализа финансовых новостей
    на основе ModernBERT, UMAP и HDBSCAN.
    \item Реализован подход к автоматическому извлечению тем (аспектов) без использования словарей и ручной разметки,
    для обеспечения возможности последующего агрегированием сентимента на уровне тем.
    \item Проведена детальная предобработка и сбор специализированного корпуса финансовых новостей, с учётом
    особенностей текстов (длина, дублирование, источники). Корпус исчисляется 1 миллиардом токенов и качественно
    спроектирован, включая богатые метаданные, что позволяет использовать его в смежных исследованиях.
    \item Разработан и формализован математический подход к семантической дедубликации, учитывающий ограниченность
    текстуального сравнения и необходимость анализа семантической уникальности.
    \item Проанализированы архитектурные преимущества ModernBERT по сравнению с BERT и FinBERT,
    а также выявлены направления дальнейшего повышения интерпретируемости моделей.
    \item Разработана высокофункциональная система для аналитики в области семантики финансовых публикаций,
    которая функционирует в режиме реального времени.
\end{enumerate}

Несмотря на достигнутые результаты, работа оставляет открытыми ряд направлений для будущих исследований.
Во-первых, необходимо обучение спроектированной архитектуры, включая экспертов и предиктивную модель,
с интеграцией временных рядов и количественными признаками для построения полноценного гибридного предиктора.
Во-вторых, перспективно развитие мультиагентной архитектуры, где модели различного уровня (например,
специализированные под различные типы документов) взаимодействуют в рамках единого пайплайна, что является
открытым полем для разработки финансового ассистента для семантического анализа публикаций. В-третьих,
дальнейшее исследование методов динамического тематического моделирования может позволить более точно
и интерпретируемо отслеживать эволюцию аспектов во времени, что критически важно в условиях изменчивого
информационного поля.

В целом, представленная работа закладывает фундамент для системного подхода к финансовому прогнозированию
с использованием больших языковых моделей в контексте теории эффективного рынка. Комбинация инженерной
реализации, математического анализа и теоретического обоснования делает предложенное решение не только
практико-ориентированным, но и значимым с точки зрения научной новизны.