\subsubsection{Mathematical Formulaion}
\label{sec:semantic_deduplication}
\sloppy  % Helps to alleviate overfull hbox warnings

Within the framework of this study, a novel deduplication approach was developed based on the analysis of the semantic content of objects.
Although in the present work the entity is a text, the method can be readily generalized to any objects that admit a vector representation
in a semantic space.

Each article is represented as a sequence of embeddings:

\begin{equation}
    x_i \subset \mathbb{R}^{t\times d}
\end{equation}

where $t$ is the number of tokens and $d$ is the dimensionality of the semantic vector space. For subsequent analysis, instead of the raw set of embeddings,
their convex hull is used, denoted as $\operatorname{CH}(x_i)$ or, for brevity, $\operatorname{CH}_i$. The uniqueness of a text is quantified by the volume
of this convex hull, $\mathrm{vol}(\operatorname{CH}_i)$.

\textbf{Accounting for the Intersections of Convex Hulls.} Direct subtraction of the intersections between $\operatorname{CH}_i$ and the hulls of other
texts may lead to multiple counting. To eliminate this issue, the inclusion--exclusion principle is applied.

Let the set of all articles except $i$ be denoted by

\begin{equation}
    \mathbb{I} = \{1, \ldots, N\} \setminus \{i\}.
\end{equation}

The intersection of $\operatorname{CH}_i$ with the hulls of articles indexed by subsets $\mathbb{J} \subseteq \mathbb{I}$ is expressed as:

\begin{equation}
    \mathrm{vol}\Big(\operatorname{CH}_i \cap \bigcap_{j \in \mathbb{J}}\operatorname{CH}_j\Big).
\end{equation}

Then, the volume of the intersection of $\operatorname{CH}_i$ with the union of the hulls of the remaining articles is computed as:

\begin{equation}\label{eq:inclusion-exclusion_substituting}
    \mathrm{vol}\left(\operatorname{CH}_i \cap \bigcup_{j \in \mathbb{I}}\operatorname{CH}_j\right)
    =
    \sum_{k=1}^{N-1} (-1)^{k-1} \sum_{\substack{\mathbb{J} \subseteq \mathbb{I} \\ |\mathbb{J}| = k}}
    \mathrm{vol}\left(\operatorname{CH}_i \cap \bigcap_{j \in \mathbb{J}}\operatorname{CH}_j\right).
\end{equation}

The uniqueness of article $i$ is defined as the fraction of its convex hull's volume that is not occupied by intersections with the hulls of other articles:

\begin{equation}
    \mu_i = \frac{\mathrm{vol}\Big( \operatorname{CH}_i \setminus \bigcup_{j \in \mathbb{I}} \operatorname{CH}_j\Big)}{\mathrm{vol}\Big( \operatorname{CH}_i\Big)}.
\end{equation}

By decomposing $\operatorname{CH}_i$ into the intersection region and its complement, we obtain:

\begin{equation}
    \mu_i = 1 - \frac{\mathrm{vol}\left(\operatorname{CH}_i \cap \bigcup_{j \in \mathbb{I}}\operatorname{CH}_j\right)}{\mathrm{vol}\left(\operatorname{CH}_i\right)}.
\end{equation}

Substituting the inclusion--exclusion formulation \ref{eq:inclusion-exclusion_substituting}, the final expression becomes:

\begin{equation}\label{eq:uniqueness}
    \mu_i = 1 - \frac{1}{\mathrm{vol}\big(\operatorname{CH}_i\big)}
    \sum_{k=1}^{N-1} (-1)^{k-1} \sum_{\substack{\mathbb{J} \subseteq \mathbb{I} \\ |\mathbb{J}| = k}}
    \mathrm{vol}\Big(\operatorname{CH}_i \cap \bigcap_{j \in \mathbb{J}}\operatorname{CH}_j\Big).
\end{equation}

The value $\mu_i \in [0, 1]$ characterizes the text's uniqueness: $\mu_i = 1$ indicates no intersections with other texts (complete uniqueness),
while $\mu_i = 0$ implies that the semantic volume of the text is entirely occupied by intersections with the hulls of other texts.

\subsubsection{Pros and Cons}
The proposed method is founded on a theoretically sound representation of text: each article is treated as the convex hull of its token embeddings.
This representation enables a precise definition of an object's semantic content, facilitates the application of the inclusion-exclusion principle
(inherited from set theory) for accurate calculation of intersection volumes, and normalizes the result so that the final uniqueness measure
lies within the interval $[0,1]$.

In addition to its theoretical rigor, the method offers several advantages:
\begin{itemize}
    \item The use of embeddings for each token allows for capturing subtle distinctions in semantic content, while aggregation via the convex hull
    yields a generalized representation of the text. This approach enables the comparison of texts of varying lengths and topics within a unified vector space.
    \item Normalization of the metric to a $[0,1]$ interval simplifies interpretation.
\end{itemize}

Conversely, the method has several notable drawbacks:
\begin{itemize}
    \item In high-dimensional spaces (e.g., 768 dimensions), the convex hull may become excessively "stretched," resulting in uninformative volume measurements,
    and its geometry may fail to accurately reflect the complex distribution of embeddings.
    \item Embeddings typically possess a complex, often non-linear structure. As the convex hull is the minimal convex set containing the data, it may enclose
    extreme points, leading to an overestimation of the occupied space and, consequently, to skewed evaluations.
    \item Since embeddings can include random noise or artifacts, the convex hull is sensitive to outliers. Minor inaccuracies in embeddings may
    disproportionately enlarge the convex hull’s volume, thus distorting the uniqueness assessment.
    \item Constructing the convex hull and computing volumes in high-dimensional spaces is computationally intensive. Moreover, applying the inclusion–exclusion
    principle to accurately compute intersections between the hulls of texts further complicates calculations, particularly with a large number of documents.
\end{itemize}

These challenges can critically affect the practical application of the method; however, some can be mitigated through engineering solutions.

The sensitivity to noise (item 3) can be partially alleviated by using the [CLS] token as the centroid of the convex hull. Introducing a coefficient
$\delta$ to normalize the "concavity" of the hull in the direction of the [CLS] embedding helps to diminish the impact of noisy components.

The computational burden (item 4) can be addressed through various strategies:
\begin{itemize}
    \item Regulating the number of inclusion–exclusion pairs (the hyperparameter $N$ in the summation) allows for an approximate evaluation
    of uniqueness while reducing computational demands.
    \item Employing DR algorithms, such as UMAP, t-SNE, or PCA, can project the original space onto a lower-dimensional one,
    substantially decreasing computational costs, though potentially at the expense of some accuracy.
    \item Approximating the volume using Monte Carlo methods offers an alternative that lessens computational load.
\end{itemize}

The method of representing semantic uniqueness of text via the convex hulls of embeddings boasts several theoretical advantages (robust normalization,
applicability of the inclusion–exclusion principle, and consistent interpretability of the result). Nevertheless, its practical deployment necessitates
addressing challenges related to high dimensionality, non-linear distributions of embeddings, and substantial computational costs. Future research may
focus on developing more robust and computationally efficient methods for assessing text uniqueness while accommodating these limitations.