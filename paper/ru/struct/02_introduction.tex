В последние годы использование анализа данных и искусственного интеллекта, в частности машинного (Machine Learning, ML)
и глубокого обучения (Deep Learning, DL), для принятия инвестиционных решений стало неотъемлемой частью стратегий многих
трейдеров, компаний и фондов. Однако современные финансовые рынки характеризуются высокой волатильностью
и высокой скоростью распространения информации, что создает значительные трудности для анализа ее влияния
на цены акций.

Такие события, как выпуск экстренных новостей, изменения в законодательстве и обзоры аналитиков, могут оказывать
как непосредственное, так и накопительное воздействие на показатели рынка. Однако традиционные подходы к анализу
данных часто игнорируют динамику этих воздействий, что приводит к низкой точности прогнозов и, как следствие,
неэффективности инвестиционных стратегий.

Чтобы конкурировать в быстро меняющейся финансовой среде, компаниям необходимо постоянно оптимизировать свои
подходы к анализу данных. Это требует разработки инструментов, которые могут не только учитывать динамический
характер информации, но и предоставлять прогнозы на основе глубокого анализа событий и их совокупного эффекта.
Данная статья отвечает на эти вызовы, предлагая методологию и технологическое решение для глубокой, интерпертируемой
и простой финансовой аналитики, основанное на глубоких нейронных сетей, а именно больших языковых моделях (LLM).

Эффективность классических ML-алгоритмов в финансовом прогнозировании доказана во многих исследованиях.
Однако DL и архитектуры для обработки естественного языка (NLP) буквально изменили правила игры после появления
трансформерной архитектуры в 2017 году \parencite{vaswani2017attention}. С тех пор LLM получили широкое распространение
и доказали свою применимость в различных прикладных задачах, в том числе и в прогнозировании стоимости активов
\parencite{Jiang2023, Halder2022, Kim2023}.

Современные исследования демонстрируют высокую эффективность LLM для решения ряда задач, связанных с оценкой и
прогнозированием активов. Тем не менее, остаются неразрешёнными вопросы интеграции LLM с классическими количественными
моделями, отсутствия достаточного количества решений с откртым исходным кодом для финансовой области и ограничений
существующих моделей для обработки длинных текстовых последовательностей (см. \hyperref[sec:models]{Раздел 1.3.1}).
В декабре 2024 года была представлена новая современная модель ModernBERT, способная обрабатывать тексты, длина которых
в 16 раз превышает возможности предыдущих архитектур \parencite{Warner2024ModernBERT, devlin2019BERT}. Эта модель
расширяет возможности анализа с отдельного заголовка или поста до целых новостных статей, пресс-релизов, транскриптов
интервью и аналитических обзоров. Несмотря на это, обработка полноценных финансовых отчётов (например, 10-Q, 10-K)
остаётся сложной задачей (см. \hyperref[sec:limitations]{Раздел 2.1}). При этом, акцент на анализе полных статей
позволяет избежать проблем, связанных с кликбейт-эффектом и недостатком контекста для принятия решения.

Языковые модели, предобученные на общих текстовых корпусах, не всегда эффективно решают задачи финансового прогнозирования
\parencite{Jiang2023}. Это объясняется уникальностью финансовой информации, богатством специальной терминологии
и жаргонизмов, что затрудняет применение моделей общего назначения. В связи с этим возникает необходимость адаптации
базовых LLM для финансовой сферы.

Таким образом, если рассматривать проблему со стороны управления, при принятии инвестиционных решений на волатильных рынках,
важно своевременно анализировать совокупное влияние различных событий (новости, законодательные изменения, аналитика и т. д.)
на динамику активов. Специалисты не способны справиться с таким объемом информации за предельно короткие сроки. С другой
стороны, отсутствие инструмента для комплексного анализа приводит к запаздывающим или неточным решениям, что снижает
эффективность инвестиционных стратегий и увеличивает риск пропуска выгодных возможностей.

Разработка инструмента, который мог бы разрешить данную ситуацию и упростить процесс принятия решений является трудоёмкой
и комплексной задачей, которую можно условно разбить на следующие этапы:

\begin{enumerate}
    \item Разработка эффективной архитектуры для LLM.
    \item Адаптация модели под специфику финансового домена.
    \item Тонкая настройка модели для решения конкретных задач.
    \item Интеграция модели в систему, работающую как с количественными, так и с качественными данными,
    включая этапы обучения, тестирования и внедрения.
\end{enumerate}

Важно отметить, что конечной целью подобного инструмента стоит автономное эффективное финансовое прогнозирование, которое
бы согло быть с легкостью интерпретировано финансовым аналитиком, который контролирует систему. Стоит уточнить, что речь
идет об эффективном финансовом прогнозировании именно с точки зрения теории эффективности рынка (EMT) \parencite{emt1970fama}.

Итак, поскольку базовая архитектура ModernBERT уже разработана, а методы прогнозирования финансовых активов при помощи нейронных
сетей достаточно изучены, настоящее исследование фокусируется разработке инструментария для повышения как эффективности, так и
интерпретируемости финансового прогнозирования. Причем исследование формулирует цель достичь повышения эффективности прогнозирование,
не за счет количественного или итеративного улучшения имеющихся методов, а за счет изучения принципиально новых областей и качетсвенного
технологического скачка, заключающегося в новой парадигме финансового прогнозирования. Так, настоящее исследование предлагает
достичь качественного скачка за счет улучшения данных используемы для прогнозирования.

Данная задача, как отмечалось ранее, комплексна и крайне трудна. Поэтому текущая работа фокусируется не на самом финансовом прогнозировании,
а на разработке инструментария, технологии и комплекса методов, которые сделают возможным в последующих исследованиях качественный рывок
в целевой задаче.

Так, на данный момент, в финансовом прогнозировании LLM используются в крайне узком спектре, так как сама технология относительно нова.
Тем не менее, уже есть первые попытки использовать языковые модели для анализа тональности новостей и интеграции тональностей в процесс
прогнозирования, причем несмотря на ограниченность ресурсов, методов и технологий данный подход зарекомендовал себя как достаточно
эффективный. В частности, модели, применяющие тональности в финансовом прогнозирвании показывают результаты выше тех, что делают
прогнозы, основываясь исключительно на количественных данных \parencite{Kim2023, Jiang2023, Halder2022}. Однако, применимость текстовой
модальности для финансового прогнозирования на данный момент крайне ограничено и новым направлением, пока еще не нашедшем применение
в финансах является аспекто-ориентированый сентиментальный анализ \parencite{SA2020taxonomy,FSA2020problems} и тематическое моделирование
\parencite{angelov2020top2vec,BERTopic2022}.

Таким образом, объектом исследования являются инвестиционные стратегии, основанные на методах искусственного интеллекта, а предметом ---
инструментарий для интеграции языковых моделей, а также методов тематического моделирования и анализа тональностей в процесс прогнозирования
стоимости активов. Цель работы заключается в создании практико-ориентированного интерпретируемого инструментария, на основе которого станет
возможным разработать систему для динамического мультимодального прогнозирования с использованием иерархического аспекто-ориентированного
сентиментального анализа. Отдельно стоит подчеркнуть, что ключевым условием предложенного решения является его интерпретируемость
в отличие от других решений, использующих глубокие нейронные сети как черный ящик.

На протяжении исследования были выполнены следующие работы:

\begin{itemize}
    \item Анализ и выбор современных и наиболее эффективных архитектур и моделей (см. Разделы \hyperref[sec:ml_algos]{1.2} и
    \hyperref[sec:models]{1.3}) для целевой задачи.
    \item Проектирование концепта прицнипиально новой системной архитектуры для эффективного финансового прогнозирования
    (см. Раздел \hyperref[sec:architecture]{3.1}).
    \item Разработка финансовой аспекто-ориентированной гибридной семантической системы (Financial Aspect-Based hybrid Semantic System, FinABYSS),
    направленной на реализацию необходимых системных комонент для интерпретируемой финансовой аналитики на мультимодальных данных
    (см. Раздел \hyperref[sec:components]{3.2}).
    \item Сбор большого и исчерпывающего корпуса финансовых публикаций размером более 15Гб, пригодного для использования в смежных финансовых исследованиях
    (см. Раздел \hyperref[sec:data_governance]{2.2}).
\end{itemize}

Все результаты данного исследования, включая код для сбора данных, обучающий код, результаты анализ данных и моделей, а также сами модели доступны
в официальном GitHub репозитории проекта\footnote{FinABYSS (Financial Aspect-Based hybrid Semantic System) [Electronic resource] //
Томин Д.В. --- 2025 --- URL: \url{https://github.com/denisalpino/FinABYSS} (Дата обращения: 26.05.2025). --- Режим доступа: по запросу.}.
Собранный корпус доступен в репозитории HuggingFace\footnote{YahooFinanceNewsRaw [Electronic resource] //
Томин Д.В. --- 2025 --- URL: \url{https://huggingface.co/datasets/denisalpino/YahooFinanceNewsRaw} (Дата обращения: 20.04.2025). --- Режим доступа: по запросу.}.