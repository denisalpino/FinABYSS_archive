\documentclass[a4paper,12pt]{extarticle}
% Глобальные поля
\usepackage[left=30mm, top=20mm, right=15mm, bottom=20mm,nohead, includefoot,footskip=35pt]{geometry}

% Язык, кодировка, шрифт
\usepackage[utf8]{inputenc}
\usepackage[T1]{fontenc}
\usepackage[english]{babel}
%\renewcommand{\rmdefault}{Tempora-TLF}
\usepackage{mathptmx}

\usepackage[
    backend=biber,
    style=gost-authoryear,
    movenames=false, % Если авторов больше 4 кеакого-то хрена меняло на название работы
    otherlangs=true
]{biblatex}
\addbibresource{bibliography.bib}
\usepackage{csquotes}

% Межстрочный интервал = 1.5pt
\usepackage{setspace}
\onehalfspacing

% Абзацный отступ = 1.25см
\usepackage{indentfirst}
\setlength\parindent{12.5mm}

% Пакет для содержания
\usepackage{tocloft}

% Команда для специальных разделов (введение, обзор литературы, etc)
% Не нумеруются в содержании, по уровню вложенности:
\newcommand{\specialsection}[1]{
    \phantomsection
    \bigskip\smallskip\hspace{-13.8mm}
    \normalfont\fontsize{12}{12}\textbf{#1}
    \par\bigskip\normalfont\normalsize
    \addcontentsline{toc}{section}{#1}
}

% Размеры заголовков разделов и подразделов
\usepackage{titlesec}
% Раздел: 12pt, добавляем слово "CHAPTER"
\titleformat{\section}
{\fontsize{12}{12}\bfseries}{
\hspace{-1.5mm}CHAPTER \thesection. \hskip-1em}{1em}{}
% Подраздел: 12pt
\titleformat{\subsection}
{\fontsize{12}{12}\bfseries}{\hspace{-0.2mm}\thesubsection}{1em}{}

% Содержание
\renewcommand{\cfttoctitlefont}{\centering\normalsize\bfseries}
\renewcommand{\cftaftertoctitle}{\hfill}

% Слово "Глава" в содержании
\renewcommand{\cftsecpresnum}{CHAPTER\space}
\newlength\mylength
\settowidth\mylength{\cftsecpresnum}
\addtolength\cftsecnumwidth{\mylength}
% Строки с точками
\renewcommand{\cftsecleader}{\cftdotfill{\cftdotsep}}
% Точки после цифр в в содержании
\renewcommand{\cftsecaftersnum}{.}
\renewcommand{\cftsubsecaftersnum}{.}
% Подровнять subsection под точку главы
% (если глав будет больше десяти, будет чуть хуже)
\setlength{\cftsubsecindent}{2.68em}
% Интервал глав
\setlength{\cftbeforesecskip}{4pt}

\renewcommand{\cftsecpagefont}{\normalfont}

% Шрифт подписи (caption) = 12pt
% (Повезло, что small как раз равен 12pt)
\usepackage[font=small,labelfont=bf]{caption}

% Пакет, который позволяет собирать один документ TeX из нескольких
\usepackage{import}

% Пакет, реализующий гиперссылки. Никакого расскрашивания
\usepackage[colorlinks=false,unicode=true]{hyperref}

\newcommand{\ITEM}{\vspace{-0.2cm}\item}
\newcommand{\MList}[1]{\par\begin{itemize}#1\end{itemize}}
\newcommand{\NList}[1]{\par\begin{enumerate}#1\end{enumerate}}


% Задаем отступ у списков равным обычному абзацному отступу (\parindent)
\usepackage{enumitem}
\setlist[itemize]{leftmargin=\dimexpr\parindent\relax}
\setlist[enumerate]{leftmargin=\dimexpr\parindent\relax}

% Нумерация только тех формул, на которые в тексте присутствует ссылка
% \usepackage{autonum}
% Правка нумерации формул: числа справа в скобках
\renewcommand{\theequation}{\arabic{equation}}

% Список литературы
% \makeatletter
% \renewenvironment{thebibliography}[1]
%     {%
%       \bigskip
%       {\noindent\normalfont\bfseries References\par}%
%       \addcontentsline{toc}{section}{References}%
%       \smallskip
%       \list{\@biblabel{\@arabic\c@enumiv}}%
%            {%
%              \settowidth\labelwidth{\@biblabel{#1}}%
%              \leftmargin\labelwidth
%              \advance\leftmargin\labelsep
%              \@openbib@code
%              \usecounter{enumiv}%
%              \let\p@enumiv\@empty
%              \renewcommand\theenumiv{\@arabic\c@enumiv}%
%            }%
%       \sloppy
%       \clubpenalty4000
%       \@clubpenalty \clubpenalty
%       \widowpenalty4000%
%       \sfcode`\.\@m%
%     }%
%     {%
%       \def\@noitemerr
%         {\@latex@warning{Empty `thebibliography' environment}}%
%       \endlist%
%     }
% \makeatother


% Пакеты по желанию (самые распространенные)
% Хитрые мат. символы
\usepackage{euscript}
% Таблицы
\usepackage{multirow}
\usepackage{longtable}
\usepackage{makecell}
% Картинки (можно встявлять даже pdf)
\usepackage[pdftex]{graphicx}

\usepackage{amsthm,amssymb,amsmath}
\usepackage{textcomp}

% Для коректной работы H в фигурах и ссылок на фигуры
\usepackage{float}

\usepackage{hyperref}

\newcommand{\argmax}{\operatornamewithlimits{argmax}}
\newcommand{\argmin}{\operatornamewithlimits{argmin}}
\DeclareMathOperator{\col}{col}

\begin{document}
    % Оглавление
    % Титульный лист диплома СПбГУ
% Временное удаление foot на titlepage
\newgeometry{left=30mm, top=20mm, right=15mm, bottom=20mm, nohead, nofoot}
\begin{titlepage}
\begin{center}

\textbf{Saint Petersburg}
\textbf{State University}

\vspace{35mm}

\textbf{\textit{\large Tomin Denis Valerievich}} \\[8mm]
% Название
\textbf{\large Bachelor Diploma Thesis}\\[3mm]
\textbf{\textit{\large Understanding the Aspect Structure of Financial Publications Using Deep Neural Networks}}

\vspace{20mm}
Level of education: Bachelor's degree\\
Direction 01.03.02 “Applied Mathematics and Informatics”\\
Basic educational program CB.5005.2015
«Management»\\
Graduated School of Management\\[25mm]


% Научный руководитель, рецензент
\begin{flushright}
\begin{minipage}[t]{0.65\textwidth}
{Supervisor:} \\
Professor, Research Center for Market Efficiency and Applied Finance, \\ Dr. Darko Vuković

\vspace{10mm}

{Peer reviewer:} \\
Senior Lecturer, Department of Finance and Accounting, \\ Vitaly Leonidovich Okulov
\end{minipage}
\end{flushright}

\vfill

{Saint Petersburg}
\par{\the\year{}}
\end{center}
\end{titlepage}
% Возвращаем настройки geometry обратно (то, что объявлено в преамбуле)
\restoregeometry
% Добавляем 1 к счетчику страниц ПОСЛЕ titlepage, чтобы исключить
% влияние titlepage environment
\addtocounter{page}{1}
    \pagebreak

    % Заявлениее о самостоятльности
    \input{}
    \pagebreak

    % Оглавленине
    \tableofcontents{}
    \pagebreak

    \specialsection{Introduction}
    <<...>>

    % Первая теоретическая глава
    \newpage
    \section{THEORETICAL OVERVIEW}
    \subsection{Problem statement}
    <<...>>

    \subsection{Methodology}
    <<...>>

    \subsubsection{Literature review}
    <<...>>

    \subsubsection{Tool review}
    <<...>>

    % Вторая практическая глава
    \newpage
    \section{PRACTICAL SOLUTION}
    \subsection{Data engineering}
    <<...>>

    \subsubsection{Data capture}
    <<...>>

    \subsubsection{Data processing}
    <<...>>

    \subsection{Architetcture}
    <<...>>

    \subsubsection{Aspect-based block}
    <<...>>

    \subsubsection{Sentimental block}
    <<...>>

    \subsubsection{Predictive block}
    <<...>>

    \subsection{Recommendations}
    <<...>>

    \newpage
    \specialsection{Conclusion}
    <<...>>

    % Библиография в cpsconf стиле
    % Аргумент {1} ниже включает переопределенный стиль с выравниванием слева
    \begin{thebibliography}{1}
        \bibitem{voc} Griffin D.W., Lim J.S. \flqq Multiband excitation vocoder\frqq. IEEE ASSP-36 (8), 1988, pp. 1223-1235.
        \bibitem{vo2} Griffin D.W., Lim J.S. \flqq Multiband  vocoder\frqq. IEEE ASSP-36 (8).
        \bibitem{vo3} Griffin D.W. \flqq Multiband \frqq. IEEE ASSP-36 (8).
    \end{thebibliography}
\end{document}